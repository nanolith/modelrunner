\section{Building the Model Runner tool}

To build the model runner tool, we need to extract the source files and the
build scripts from the document.  The extraction script is the first script that
we need as part of the build process.  It lists all of the file sections in this
document, and filters them between executable scripts and regular source files.
We also want to be able to run the unit tests and model checks for this tool, so
the related sources, scripts, and definitions will be pulled out for these
stages as well.

#[language=sh]
<<FILE:scripts/extract_files.sh>>=#!/bin/sh
    filename=../modelrunner.tex
    <<extract-shell-scripts>>
    <<extract-everything-else>>
>>@<<

To extract shell scripts, we just run \verb/mintangle/, passing it the \verb/-L/
option, which instructs it to list all files available for extract. This is
passed through \verb/egrep/ to filter out just the shell scripts. We iterate
through this list, extracting each by specifying them with the \verb/-r/, or
root, option, and then we make each executable in turn.

#[language=sh]
<<extract-shell-scripts>>=
    for n in `mintangle -L $filename | egrep ".sh$"`; do
        mintangle -r $n $filename
        chmod +x $n
    done #extract shell scripts from $filename
>>@<<

The steps for extracting everything else are the same, except that we use the
\verb/-v/ option with \verb/egrep/ to \emph{exclude} shell scripts, and we don't
mark these files as executable.

#[language=sh]
<<extract-everything-else>>=
    for n in `mintangle -L $filename | egrep -v ".sh$"`; do
        mintangle -r $n $filename
    done
>>@<<

\subsection{Compilation}

Because all of the source files are extracted at once when we run the
extraction, we can't take advantage of incremental builds without the use of a
custom build system. That is beyond the scope of this project, so we will just
use a simple shell script for performing compilation.  There are two executables
produced as part of the compilation process: the model runner and the unit test
runner. To build everything, the build shell script gathers all source files,
builds them, and then links the two executables.

#[language=sh]
<<FILE:scripts/build.sh>>=#!/bin/sh
sourcefiles=`find src -name "*.c" -print`
testfiles=`find test -name "*.c" -print`
if [ "$CC" == "" ]; then
  export CC=cc
fi #set $CC to a sane value
>>@<<

The script first makes a list of all C source files in the \verb/src/ and
\verb/test/ subdirectories.  Next, it sets the \verb/CC/ variable if it is not
already set.

We can then build all of the object files for both source and test. We keep
these build steps separate in case we need to add options to one or the other.

#[language=sh]
<<FILE:scripts/build.sh>>=
CFLAGS="$CFLAGS `pkg-config --cflags rcpr`"
#build source files
for n in $sourcefiles; do
    $CC $CFLAGS -I include -c -o $n.o $n
    if [ $? -ne 0 ]; then
        echo "Compilation failed."
        exit 1
    fi
done

#build test files
for n in $testfiles; do
    $CC $CFLAGS -I include -c -o $n.o $n
    if [ $? -ne 0 ]; then
        echo "Compilation failed."
        exit 1
    fi
done #$
>>@<<

Once this step is done, we can scan the directories again to pick up all of the
object files. There are three categories here: objects for the main executable,
objects for running tests, and objects for including in the tests. Later on, we
will add some test coverage options to this last category, but for now, we are
just filtering out the model runner's main method so we can substitute the test
runner's main method.

#[language=sh]
<<FILE:scripts/build.sh>>=
objects=`find src -name "*.o" -print`
objects_except_main=`find src -name "*.o" -print | grep -v main.c.o`
testobjects=`find test -name "*.o" -print`
LDFLAGS="$LDFLAGS `pkg-config --libs rcpr`"
#link modelrunner
$CC -o modelrunner $objects $LDFLAGS
if [ $? -ne 0 ]; then
    echo "Link failed."
    exit 1
fi

#link testmodelrunner
$CC -o testmodelrunner $testobjects $objects_except_main $LDFLAGS
if [ $? -ne 0 ]; then
    echo "Test link failed."
    exit 1
fi #$
>>@<<

Before running the test runner, we need to extract all test inputs and clean up
all test outputs from previous test runs.  The \verb/minextract/ utility has a
\verb/-L/ option which lists all of the test sections. We iterate thorugh this
list, extracting all of the input files and deleting all of the output files.

#[language=sh]
<<FILE:scripts/build.sh>>=
filename=../modelrunner.tex
tests=`minextract -L $filename`

for n in $tests; do
    rm -f $n.output
    minextract -S $n $filename
done #$
>>@<<

Finally, we always execute the unit tests as part of the build process.

#[language=sh]
<<FILE:scripts/build.sh>>=
./testmodelrunner
if [ $? -ne 0 ]; then
    echo "Unit testing failed."
    exit 1
fi #$
>>@<<

\subsection{Building the Document}

To build this document, we just need to run it through \verb/minweave/ and
\verb/pdflatex/.  We use the \verb/-T/ argument with \verb/minweave/ to pass it
the template we use for this document.

#[language=sh]
<<FILE:scripts/build_document.sh>>=#!/bin/sh
minweave -T ../template.tex ../modelrunner.tex
pdflatex modelrunner.tex.tex
pdflatex modelrunner.tex.tex
>>@<<
