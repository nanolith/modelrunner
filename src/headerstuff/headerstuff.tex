\section{Header Related Stuff}

This is a short section covering some preliminaries we need to build header
files.

First, in each header, there is a preamble.  This preamble includes a pragma
that tells the compiler to only include this header once.  This way, if it is
included multiple times through the list of header dependencies, its contents
will only be read by the preprocessor once.

#[language=c]
<<header-preamble>>=
#pragma once
>>@<<

Language compatibility is another issue.  The C++ language in particular is not
fully backwards compatible with C.  One significant incompatibility has to do
with name mangling.  In C++, it is necessary to mangle names in order to add
concepts to the object file such as namespaces, class membership, and function
polymorphism.  Neither C nor the linker understand these concepts, so C++
mangles the names of functions.  If a C++ program includes one of our headers,
we don't want it to make external symbol references to externals we define, or
it will complain at link time that it can't find our definitions.  So, we
incoporate a language feature of C++ that lets it know that all of the
definitions in our header are C definitions.  We place this, along with the
preamble, in our header prologue.

#[language=c]
<<header-prologue>>=
<<header-preamble>>
#ifdef   __cplusplus
extern "C" {
#endif /*__cplusplus*/
>>@<<

The extern definition is wrapped in a macro so that it is only interpreted by a
C++ compiler.  If we use a C compiler, it will not understand this extern
setting.  The epilogue for our header closes this extern definition.

#[language=C]
<<header-epilogue>>=
#ifdef   __cplusplus
} /* extern "C" */
#endif /*__cplusplus*/
>>@<<

Our headers can now make use of the \verb/header-prologue/ and
\verb/header-epilogue/ macros to automatically include these definitions.
