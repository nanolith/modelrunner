\section{Literate Unit Testing}

The first challenge we encounter when writing a modern literate application is
finding a way to incorporate unit tests into the narrative. Minweb has a
facility for capturing test values, running these values through a test
framework, and pulling the outputs back into the literate document.  We will
make use of these.  However, we need to develop a way to read these test values,
input them into unit tests, and capture the output.  We don't want to go through
a bunch of manual setup for each test, as this would detract from the narrative.
So, as much as possible, we want to automate this marshalling process so we can
focus just on the unit tests themselves. Ideally, we want to write these tests
in the same language that we use for other parts of the application.  In this
case, we are using modern C for the application.  Thus, we will use C for the
tests.

Before we get into the design and implementation of marshalling facilities,
let's start by defining our unit test runner.

\subsection{Unit Test Runner}

The unit test runner iterates through an array of unit test suites, each of
which are arrays of unit test cases.  Unit test cases are functions taking no
parameters and returning a Boolean value. If the function returns \verb/true/,
then the test passes. If the function returns \verb/false/, then the test fails.
To provide reporting to the user, the array actually contains two items: the
name of the test and the test function pointer. In order to simplify test
registration, we will provide a simple macro that expands to both of these items
when given the name of a test.

Let's start with the test header.  This header will start with the definition of
our unit test function type.  Then, it will declare the structure that we use
for holding unit test cases.

#[language=c]
<<FILE:include/modelrunner/test.h>>=
<<header-prologue>>

#include <stdbool.h>
#include <stddef.h>

typedef bool (*unit_test_function)();

typedef struct unit_test_case unit_test_case;
struct unit_test_case
{
    const char* name;
    unit_test_function function;
};
>>@<<

The \verb/unit_test_function/ type defines a function pointer that takes no
arguments and returns a Boolean (\verb/bool/) value indicating whether the test
passed (\verb/true/) or the test failed (\verb/false/). The
\verb/unit_test_case/ type is a structure containing the name of the unit test
and the function to call to run this unit test.

Next, we define the \verb/unit_test_suite/ type that we use for defining test
suites.

#[language=c]
<<FILE:include/modelrunner/test.h>>=
typedef struct unit_test_suite unit_test_suite;
struct unit_test_suite
{
    const char* name;
    unit_test_case* array;
    size_t entries;
};
>>@<<

The \verb/unit_test_suite/ has a name, an array, and a number of entries in the
suite. Now, we can create some helper macros that will make life easier for
defining test suites and test cases. The first macro, \verb/TEST_SUITE/, will
expand to the name of the test suite, the test suite array, and the number of
entries in the test suite. The \verb/TEST_SUITE_DECL/ macro provides us a helper
for declaring the test suite array.

#[language=c]
<<FILE:include/modelrunner/test.h>>=
#define TEST_SUITE(suite) \
    { .name = #suite, .array = suite ## _array, \
      .entries = sizeof( suite ## _array ) / sizeof(unit_test_case) },
#define TEST_SUITE_DECL(suite) \
    extern unit_test_case suite ## _array [];
>>@<<

This expands the provided suite to a string representation of its name, then an
array reference, and then a computed size of this array.  We'll see in a moment
how the test runner can use this information. We also define a
\verb/TEST_SUITE_BEGIN/ macro and a \verb/TEST_SUITE_END/ macro to automate the
creation of an array for holding test cases in a test suite.

#[language=c]
<<FILE:include/modelrunner/test.h>>=
#define TEST_SUITE_BEGIN(suite) \
    unit_test_case suite ## _array [] = \
    {
#define TEST_SUITE_END() };
>>@<<

We make a similar expansion as \verb/TEST_SUITE/ for \verb/TEST/. Likewise, the
\verb/TEST_DECL/ provides us with a way to declare the unit test function.

#[language=c]
<<FILE:include/modelrunner/test.h>>=
#define TEST(testcase) \
    { .name = # testcase, .function=& testcase },
#define TEST_DECL(testcase) \
    bool testcase();
>>@<<

In this expansion, we expand the test case to a string for its name, and then
take the address of the test case to get a function pointer, which is assigned
to the function part of this definition. We can combine all of this to set up a
test runner, but first, let's complete the test header with its epilogue.

#[language=c]
<<FILE:include/modelrunner/test.h>>=
<<header-epilogue>>
>>@<<

The unit test runner includes the \verb/main/ entry point for our unit tests. It
iterates through each test suite, and then iterates through each unit test in
each suite. For each unit test, it runs the test. If it passes, it emits a dot.
If it fails, it emits an exclamation point and the name of the failing test.

#[language=c]
<<FILE:test/test_runner.c>>=
#include <modelrunner/test.h>
#include <stdio.h>

<<test-suite-definition>>
<<test-runner-helpers>>
<<test-runner-main>>
>>@<<

The first thing we do with the test runner is pull in the \verb/test-suites/
macro.  More on this in a moment.

Next, we define a \verb/main/ entry point that sets up the test suite. The main
entry point computes some initial tallies, then runs all of the tests.

#[language=c]
<<test-runner-main>>=
int main(int argc, char* argv[])
{
    size_t tests = sizeof(suite) / sizeof(unit_test_suite);
    size_t total_suites = 0;
    size_t total_tests = 0;
    size_t pass_count = 0;
    size_t fail_count = 0;

    for (size_t i = 0; i < tests; ++i, ++total_suites)
    {
        printf("%s [%zu]: ", suite[i].name, suite[i].entries);

        for (size_t j = 0; j < suite[i].entries; ++j, ++total_tests)
        {
            bool testpass = suite[i].array[j].function();
            if (testpass)
            {
                printf(".");
                ++pass_count;
            }
            else
            {
                printf("!%s ", suite[i].array[j].name);
                ++fail_count;
            }
        }

        printf("\n");
    }
>>@<<

The outer loop in \verb/main/ iterates through the test suites. The name of the
suite along with the total number of tests is output. Then, the inner loop is
run over each test. For each test, it is executed. If it passes, a single dot is
output. If it fails, then an exclamation point along with the name of the
failing test is output.  Finally, we output the test results.  If all tests
passed, then we return $0$. If any test fails, we return $1$.

#[language=c]
<<test-runner-main>>=
    printf(
        "\nRan %zu test%s in %zu suite%s: %zu test%s passed and "
        "%zu test%s failed.\n",
        total_tests, plural(total_tests), total_suites, plural(total_suites),
        pass_count, plural(pass_count), fail_count, plural(fail_count));

    if (fail_count > 0)
        return 1;
    else
        return 0;
}
>>@<<

We make use of a helper that we call \verb/plural/ in the above tabulation. This
just makes the English read a little better.  The \verb/plural/ macro expands to
a simple ternary operator based on comparing the argument with one. If the
argument is exactly one, then the singular form is used. Otherwise, the plural
form is used.

#[language=c]
<<test-runner-helpers>>=
#define plural(val) ((val == 1) ? "" : "s")
>>@<<

The \verb/test-suite-definition/ is where the first bit of magic happens. Since
we are using minweb as our literate programming processor, we can take advantage
of minweb macros to append test suites on the fly in the rest of the document,
as we add them.  This definition sets up the possibility.  In the definition, we
define the \verb/suite/ array and reference a \verb/test-suite/ macro, which
expands to each of the entries in this array.  When we define a new test suite,
we simply append to this macro.

Prior to the test suites, we provide three more macros: \verb/test-decls/,
\verb/test-suite-decls/, and \verb/test-suite-arrays/.  These three macros will
hold the unit test declarations, the test suite array declarations, and the test
suite arrays themselves.  These all need to occur before the \verb/suite/ array
for dependency reasons.

#[language=c]
<<test-suite-definition>>=

<<test-decls>>
<<test-suite-decls>>
<<test-suite-arrays>>

unit_test_suite suite[] =
{
    <<test-suite>>
};
>>@<<

Finally, we define a macro, \verb/test-suite-arrays/, which is where we will
store the array definitions for each test suite.  These will all be loaded in
the test runner at compile time, which links everything together nicely.

We can test the test runner by creating a simple test suite and test in the
suite to verify that this mechanism works.  First, we add a new test suite,
which we call, \verb/metatest/. We need to add a declaration to the
\verb/test-suite-decls/ macro and register it in the \verb/test-suite/ macro.

#[language=c]
<<test-suite-decls>>=
TEST_SUITE_DECL(metatest)
>>@<<

#[language=c]
<<test-suite>>=
TEST_SUITE(metatest)
>>@<<

For this test suite to execute, we use the \verb/TEST_SUITE_BEGIN/ and
\verb/TEST_SUITE_END/ macros. Inside of these macros, we place a macro for
holding our unit tests.

#[language=c]
<<test-suite-arrays>>=
TEST_SUITE_BEGIN(metatest)
<<metatest-tests>>
TEST_SUITE_END()
>>@<<

<<SECTION:metatest_simple>>=
Here, we define a \verb/metatest-tests/ macro, which we can use for appending
tests to our metatest suite. For now, we will append a single unit test,
\verb/metatest_simple/. In the next section, we will concentrate on marshalling
data from our document to the unit tests, but for now, we will include a very
simple pass or fail metric.  We can do this by embedding our interaction in a
special section, as done in the source for this document. Then, we reference an
output variable. Our test is currently \colorbox{black}.
>>@<<

#[language=c]
<<test-decls>>=
TEST_DECL(metatest_simple)
>>@<<

#[language=c]
<<metatest-tests>>=
TEST(metatest_simple)
>>@<<

We declare the unit test in \verb/test-decls/, which makes this symbol
visible to the suite. Then, we add this test to the suite by adding it to the
\verb/metatest-tests/ macro that we defined above. This ensures that the test is
executed as part of our suite.  Now, we can implement a simple test of our test
runner and testing framework.  The \verb/metatest_simple/ test writes to an
output file that is incorporated into the above paragraph. The last line of the
paragraph displays the test status. For now, we just hard code a "passing"
status and return \verb/true/ to let the test runner know that our unit test
passed. In the next section, we will improve upon this marshalling and I/O to
automate this interaction.

#[language=c]
<<FILE:test/metatest/simple.c>>=
#include <modelrunner/test.h>
#include <stdio.h>

bool metatest_simple()
{
    FILE* f = fopen("metatest_simple.output", "w");
    if (NULL == f)
    {
        return false;
    }

    fprintf(f, "testresult=\\textcolor{green}{passing}\n");
    fclose(f);
    return true;
}
>>@<<

\subsection{Marshalling Unit Test Values}

Not yet implemented.
