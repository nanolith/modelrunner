\section{Lexical Analysis}

The first step of building our scenario language is to reduce the input to a
series of tokens. Each token will represent a particular type of input, which we
will then use when we construct our parser. There are six kinds of tokens:
keywords, structural tokens, assignments, identifiers, globs, and special.
Keywords are specific terms, like \verb/rootdir/, \verb/set/, \verb/set+/,
\verb/sourcefiles/, \verb/scenario/, or \verb/require/. These terms have
meanings and help to give our scenarios structure. Structural tokens, such as
\verb/{/ or \verb/}/, are used to begin or end sections. Identifiers are a
restricted form of globs that always start with an alphabet character or
underscore, then contain any number of alphanumeric or underscore characters.
These are case sensitive, and are used for naming certain sections. Assignments
are any glob that contain an equal sign. Then, globs cover any other token.
Whitespace separates tokens from each other. When whitespace is needed in a
glob, a double quote can be used to surround the glob. This double quote will be
stripped from the glob, but any quote symbols within the glob can be escaped
with a backslash.  These lexical rules aren't perfect, but they should cover a
useful subset of scenarios and filename descriptions.  The purpose of this
language isn't to support any possible filename, but to make it easier to
represent model checking scenarios within a literate programming document.  The
final token type is the special token type.

We also want to track filename, line, and column information in our lexical
analyzer.  This data will be useful for emitting meaningful errors in the
parser. Since we are dealing with the output of a literate tool, we will add a
special token to allow this tool to manipulate our filename and line
information. In order to ensure that this remains compatible with C, we will
use the same preprocessing directive used in C for this purpose: \verb/#line/.
The \verb/#line/ preprocessing directive in C takes one mandatory argument --
the line number of the line following this line -- and one optional argument.
The optional argument is the filename. This directive requires a newline
immediately after the last argument. This newline is important in order to
continue the token stream where it left off.  The lexer treats this as
out-of-band data and handles it internally. The user of the lexer -- typically
the parser -- can't read these tokens. The lexer takes care of them and elides
them from the output.

We will start our lexer header with the enumeration of token types. Then, we
will describe types needed by the lexer, and finally, we will describe functions
needed by the lexer.  Each of these sections will be filled in as we touch on
concepts relating to them.

#[language=c]
<<FILE:include/modelrunner/lexer.h>>=
<<header-prologue>>
#include <rcpr/psock.h>
#include <rcpr/resource.h>
#include <rcpr/resource/protected.h>
<<lexer-token-enumeration>>
<<lexer-types>>
<<lexer-functions>>
<<header-epilogue>>
>>@<<

The token enumeration will hold each of the tokens we define.

#[language=c]
<<lexer-token-enumeration>>=
enum lexer_token
{
    <<lexer-token-enum>>
};
>>@<<

\subsection{Lexer Basics}

To parse these tokens, we will first need to build a lexer instance. This
instance will be based on the type \verb/lexer/, which we can define as a
\verb/resource/. At a minimum, the resource will need an allocator reference for
handling cleanup. Then, we can add additional members to this structure as
needed, by appending to the \verb/lexer-struct-members/ macro.

#[language=c]
<<lexer-types>>=
typedef struct lexer lexer;
struct lexer
{
    RCPR_SYM(resource) hdr;
    RCPR_SYM(allocator)* alloc;
    <<lexer-struct-members>>
};
>>@<<

The lexer will be initialized with the variables \verb/curline/, \verb/curcol/,
\verb/startline/, \verb/startcol/, \verb/endline/, and \verb/endcol/, which will
track the start and end position of tokens, as well as the running position. It
will need \verb/token_buffer/, \verb/token_buffer_size/, and
\verb/token_buffer_offset/ variables to hold the current dynamic token buffer,
its total size, and the current offset into which we are appending data. We will
also include a \verb/filename/, which caches the name of the current input
filename.

#[language=c]
<<lexer-struct-members>>=
    char* filename;
    size_t curline;
    size_t curcol;
    size_t startline;
    size_t startcol;
    size_t endline;
    size_t endcol;
    char* token_buffer;
    size_t token_buffer_size;
    size_t token_buffer_offset;
>>@<<

In order for the lexer to work, it has to pull data from a stream.  This stream
needs to come from somewhere.  The \verb/RCPR/ library has an interface known as
\verb/psock/, which includes a \verb/psock_read/ method. We can make use of this
interface to read data both from files and from arbitrary streams, such as
string streams.  We will use the latter for testing.  We will also need a
``putback'' facility, which allows us to read ahead in the stream and put this
value back if we aren't going to consume it.  This look-ahead allows us to
analyze more complex input.  Thus, we will add a \verb/stream/ (\verb/psock/)
and a \verb/putback/ buffer with \verb/putback_offset/.  We will size the latter
buffer so that we can manage as much lookahead as is required.

#[language=c]
<<lexer-struct-members>>=
    RCPR_SYM(psock)* stream;
    char putback[16];
    size_t putback_offset;
>>@<<

\subsection{Creating a Lexical Analyzer}

Before we can parse individual tokens, we need code to create the lexical
analyzer resource and to release this resource when it is no longer needed. We
will create a function called \verb/lexer_create/ to create a new lexer
instance, and \verb/lexer_resource_handle/ to return the resource handle for
this instance.  This resource handle can then be passed to
\verb/resource_release/ to release it when the caller no longer needs it.

#[language=c]
<<lexer-functions>>=
status lexer_create(
    lexer** l, RCPR_SYM(allocator)* a, RCPR_SYM(psock)* s,
    const char* filename);
RCPR_SYM(resource)* lexer_resource_handle(lexer* l);
>>@<<

The \verb/lexer_create/ function takes an allocator instance and a psock
instance. On success, it assigns a new lexer instance to the \verb/l/ pointer
pointer, and returns \verb/STATUS_SUCCESS/.  The \verb/lexer_resource_handle/
function returns the resource handle for this lexer instance.  Its definition is
as follows.

<<FILE:src/lexer/lexer_resource_handle.c>>=
#include <modelrunner/lexer.h>

RCPR_IMPORT_resource;

resource* lexer_resource_handle(lexer* l)
{
    return &l->hdr;
}
>>@<<

To test that this resource handle function works correctly, we first need to
create a test suite for the lexical analyzer.  We do this by extending the test
suite section and the test suite arrays section mentioned in the test harness
section with a new test suite.

#[language=c]
<<test-suite>>=
TEST_SUITE(lexer)
>>@<<

#[language=c]
<<test-suite-arrays>>=
TEST_SUITE_BEGIN(lexer)
<<lexer-tests>>
TEST_SUITE_END()
>>@<<

<<SECTION:test_lexer_resource_handle>>=
Now, we can define a unit test for the resource handle function. This test will
create a lexer instance, get its resource handle, compare it to the original
pointer, and then release this handle. This just ensures that simple creation
and tear-down work correctly.  This test ran with
\colorbox{black}.
>>@<<

To add this test, we need to add it to our list of test declarations and to our
list of tests to be executed in the test suite.

#[language=c]
<<test-decls>>=
TEST_DECL(test_lexer_resource_handle)
>>@<<

#[language=c]
<<lexer-tests>>=
TEST(test_lexer_resource_handle)
>>@<<

The test itself works exactly as described above.

#[language=c]
<<FILE:test/lexer/test_lexer_resource_handle.c>>=
#include <modelrunner/lexer.h>
#include <modelrunner/test.h>
#include <string.h>

RCPR_IMPORT_allocator;
RCPR_IMPORT_psock;
RCPR_IMPORT_resource;

bool test_lexer_resource_handle()
{
    lexer* l = NULL;
    psock* sock = NULL;
    resource* handle = NULL;
    const char* inp = "test input";

    TEST_BEGIN(test_lexer_resource_handle);
        TEST_ASSERT(
            STATUS_SUCCESS
                == psock_create_from_buffer(
                        &sock, alloc, inp, strlen(inp)));
        TEST_ASSERT(STATUS_SUCCESS == lexer_create(&l, alloc, sock, "test.x"));
        handle = lexer_resource_handle(l);
        if ((void*)l == (void*)handle)
        {
            WRITE_SUCCESS("testresult", 0);
        }
        else
        {
            WRITE_FAILURE("testresult", 0);
            TEST_FAILURE();
        }
        TEST_ASSERT(STATUS_SUCCESS == resource_release(handle));
    TEST_END();
}
>>@<<

The \verb/lexer_create/ function allocates memory for the lexer instance,
initializes it as a resource, sets the initial values, creates a token buffer,
and returns the whole mess to the caller. It also handles back-tracking and
cleanup should any of these steps fail.

#[language=c]
<<FILE:src/lexer/lexer_create.c>>=
#include <modelrunner/lexer.h>
#include <string.h>

RCPR_IMPORT_allocator;
RCPR_IMPORT_psock;
RCPR_IMPORT_resource;

<<lexer_create_decls>>

status lexer_create(lexer** l, allocator* a, psock* s, const char* filename)
{
    <<lexer_create_localvars>>
    <<lexer_create_allocation>>
    <<lexer_create_resource_init>>
    <<lexer_create_set_values>>
    <<lexer_create_create_token_buffer>>
    <<lexer_create_return_to_caller>>
    <<lexer_create_cleanup>>
}

<<lexer_create_defs>>
>>@<<

We'll start with memory allocation for the instance.  This is pretty
straight-forward: we use the \verb/tmp/ variable to hold the instance as we
build it, so we don't change the \verb/l/ parameter until everything succeeds.
After allocation, we clear the memory to ensure that everything is in a known
good state.

#[language=c]
<<lexer_create_localvars>>=
    status retval, release_retval;
    lexer* tmp;
>>@<<

#[language=c]
<<lexer_create_allocation>>=
    retval = allocator_allocate(a, (void**)&tmp, sizeof(lexer));
    if (STATUS_SUCCESS != retval)
    {
        goto done;
    }
>>@<<

If allocation fails, we exit the function, returning the failure code.

#[language=c]
<<lexer_create_cleanup>>=
done:
    return retval;
>>@<<

The allocator does not guarantee that memory is cleared, so we clear it.

#[language=c]
<<lexer_create_allocation>>=
    memset(tmp, 0, sizeof(*tmp));
>>@<<

The lexer instance is a resource. This means that it has a special resource
release method, which is called when the caller releases its resource handle.
This release method is called \verb/lexer_resource_release/.

#[language=c]
<<lexer_create_decls>>=
static status lexer_resource_release(resource* r);
>>@<<

The this function effectively tears down this instance, releasing all resources
that it owns.  Working backward, we can clean up all of these resources by
examining the structure.  The structure has a \verb/filename/, which is
duplicated based on either the initial input parameter or as part of a list
preprocessor override. It has a token buffer, which is written to during lexical
analysis. It has a \verb/psock/ stream from which it reads. We can tear down
each of these resources in turn, and accumulate errors to return to the caller,
if there are any errors.

#[language=c]
<<lexer_create_defs>>=
static status lexer_resource_release(resource* r)
{
    status release_retval;
    status retval = STATUS_SUCCESS;
    lexer* l = (lexer*)r;

    allocator* a = l->alloc;

    release_retval = allocator_reclaim(a, l->filename);
    if (STATUS_SUCCESS != release_retval) retval = release_retval;
    release_retval = allocator_reclaim(a, l->token_buffer);
    if (STATUS_SUCCESS != release_retval) retval = release_retval;
    release_retval = resource_release(psock_resource_handle(l->stream));
    if (STATUS_SUCCESS != release_retval) retval = release_retval;
    release_retval = allocator_reclaim(a, l);
    if (STATUS_SUCCESS != release_retval) retval = release_retval;

    return retval;
}
>>@<<

With the resource release method defined, we can get back to using it in order
to initialize the lexer resource in \verb/lexer_create/.

#[language=c]
<<lexer_create_resource_init>>=
    resource_init(&tmp->hdr, &lexer_resource_release);
>>@<<

We can now set the initial values for the lexer. We cache the allocator used to
create this instance, and we duplicate the filename from the input parameter.

#[language=c]
<<lexer_create_set_values>>=
    tmp->alloc = a;
    tmp->filename = strdup(filename);
>>@<<

String duplication can fail, and if it does, we want to clean up the lexer
structure. At this point, the resource has not been fully created, so we must
clean it up piecemeal.  In this case, we can just reclaim the memory we
allocated for the instance on failure.

#[language=c]
<<lexer_create_set_values>>=
    if (NULL == tmp->filename)
    {
        retval = ERROR_GENERAL_OUT_OF_MEMORY;
        goto reclaim_lexer_instance;
    }
>>@<<

#[language=c]
<<lexer_create_cleanup>>=
reclaim_lexer_instance:
    release_retval = allocator_reclaim(a, tmp);
    if (STATUS_SUCCESS != release_retval)
    {
        retval = release_retval;
    }

    goto done;
>>@<<

The allocator reclaim could fail, at which point, we want to make sure that the
error code is bubbled up. Succeed or fail, we jump to \verb/done/ in order to
exit the function.  Each of the cleanup steps will be chained together like this
to ensure that the piecemeal cleanup is done as an unwinding action.

The current line, or \verb/curline/, is set to $1$. All other positions are
set to $0$. As the lexer reads input, it will always start on the first line,
but will count columns as it encounters characters. Since this structure was
cleared out, all values are $0$ or \verb/NULL/ by default.

#[language=c]
<<lexer_create_set_values>>=
    tmp->curline = 1;
>>@<<

On success, the lexer instance takes ownership of the \verb/psock/ stream. We
set the value here.

#[language=c]
<<lexer_create_set_values>>=
    tmp->stream = s;
>>@<<

By default, we set the maximum size of the token buffer to $16$ kilobytes.
This should be more than sufficient size for any token we will encounter. If we
encounter a token larger than this, we will simply truncate it.

#[language=c]
<<lexer_create_decls>>=
#define TOKEN_BUFFER_MAX_SIZE   16384
>>@<<

We then allocate the token buffer. If allocation fails, we will unwind the
creation of the lexer instance.

#[language=c]
<<lexer_create_create_token_buffer>>=
    retval =
        allocator_allocate(
            a, (void**)&tmp->token_buffer, TOKEN_BUFFER_MAX_SIZE);
    if (STATUS_SUCCESS != retval)
    {
        goto cleanup_filename;
    }
>>@<<

The next step of the unwind is to reclaim memory for the filename. Since we are
using the malloc allocator, we can just use the allocator reclaim method.

#[language=c]
<<lexer_create_cleanup>>=
cleanup_filename:
    release_retval = allocator_reclaim(a, tmp->filename);
    if (STATUS_SUCCESS != release_retval)
    {
        retval = release_retval;
    }

    goto reclaim_lexer_instance;
>>@<<

The next step in the unwind is to jump to \verb/reclaim_lexer_instance/, which
continues cleanup with our previous cleanup step. If all steps to this point
have succeeded, then we can return the temporary instance to the caller. At this
point, the caller can call \verb/resource_release/ on the lexer's resource
handle in order to release this resource when the caller no longer needs it.
Also, the instance invariants have all been set, so we can treat this instance
as a "live" instance. So, we assign this instance to the caller's output
variable and return a successful status code, to let the caller know that the
creation process was successful.

#[language=c]
<<lexer_create_return_to_caller>>=
    *l = tmp;
    retval = STATUS_SUCCESS;
    goto done;
>>@<<

\subsection{Line Preprocessor Statements}

The first of the supported tokens will be the preprocessor line token.  It
starts with a newline, then has a hash character, any number of whitespace
characters, and a sequence of ``line'', followed by at least one argument, then
a newline.  It will \emph{not} be emitted to the caller. Instead, it will update
the internal state of the lexer to change the display of the line number and
filename.

We will first define a lexical token for lines, which we will use just as an
internal place holder.

#[language=c]
<<lexer-token-enum>>=
    LEXER_TOKEN_INTERNAL_PREPROCESSOR_LINE,
>>@<<
