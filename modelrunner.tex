\title{Model Runner Design and Implementation}

\section{Introduction}

This document covers the design and implementation of the Model Runner tool.
This document is a literate programming document, meaning that the source code
is interwoven into the document. This goes a bit further than other literate
programming documents because the unit tests and model checks are also included.
We not only talk about the design in this document, but we test it and prove
that, to the best of our knowledge and within the limitations of our tooling and
support libraries, it is correct.

Model Runner is a tool that runs a suite of model checks for a given software
package. On execution, this tool builds a tree of modules, model checking
scenarios, header files, and source files. It then optionally reconciles this
tree against a cache of previous model checking results, and executes any stale
model checks.  The results are displayed to the user.  Successful model checks
are optionally cached so they don't have to be re-run until conditions change.
Finally, it supports extracting settings that can go into a report for other
literate programming projects.

The first appendix covers how to extract and build the source code for this
tool.  The remaining sections will cover lexical analysis, parsing, syntax tree
building, semantic analysis, synthesis, execution, caching, and the
implementation of the main executable.  Interspersed in these sections are unit
tests and model checks.  The overall document is fairly long.  For ease of
development, it is organized into separate pieces that each cover a different
small aspect of the entire software project. This is done to ensure that
checkins make sense and that the development process is more natural.

#[include=src/lexer/lexer.tex]
#[include=src/parser/parser.tex]
#[include=src/ast/ast.tex]
#[include=src/analysis/analysis.tex]
#[include=src/synthesis/synthesis.tex]
#[include=src/execution/execution.tex]
#[include=src/caching/caching.tex]
#[include=src/modelrunner/main.tex]

\appendix

#[include=scripts/building.tex]
